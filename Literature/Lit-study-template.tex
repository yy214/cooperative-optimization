\documentclass[10pt,english]{article}
\usepackage{babel,amsmath,amsfonts,amssymb,color,fancybox}
\usepackage[latin1]{inputenc}
\usepackage[a4paper, inner=2.5cm, outer=2.5cm, top=2.0cm, bottom=2cm, bindingoffset=0cm]{geometry}
\usepackage{graphicx}

%\usepackage{times}                  % stile delle sections

\newcommand{\reals}{\mathbf{R}}
\parindent=0pt \parskip=3pt
%
\title{Review of Research Paper number ??}
\date{version: \today}%Meeting held on: April the 2nd}
\author{author 1, author 2, author 3}
\begin{document}

\maketitle
%\small

\section{Introduction}

\subsection{Summary of the paper}

Summarise the paper in your own words, the context, the main results, and which algorithms of the course it improves on (if any). [10 lines] 

\subsection{Problem formulation}

Introduce and report the problem that the paper presents in mathematical terms, {\bf by using the notation of the lecture notes.} [1 page max]

\subsection{Assumptions}

Introduce the main assumptions that the paper needs to derive the theoretical results. [1 page max]

\section{Result: theory and practice}

\subsection{Theory}

Describe the main theorems of the paper and the results therein in mathematical terms, {\bf by using the notation of the lecture notes.} Interpret the results [1 page max]

\subsection{Practice}

Code the main algorithm of the paper and apply it to the kernel setting of your python project. Compare the result with the most similar algorithm from the course. Comment on the results. [1 page max] [Send a link/file to the python code].

\section{Conclusion}

Draw some conclusion to put the paper in perspective with the course. [10 lines]

\end{document}



